\documentclass[a4paper,titlepage,11pt,twosides,floatssmall]{mwrep}
\usepackage[left=2.5cm,right=2.5cm,top=2.5cm,bottom=2.5cm]{geometry}
\usepackage[OT1]{fontenc}
\usepackage{polski}
\usepackage{amsmath}
\usepackage{xr}
\usepackage{mathtools}
\usepackage{amsfonts}
\usepackage{multirow}
\usepackage{svg}
\usepackage{amsmath}
\usepackage{makecell}
\usepackage{amssymb}
\usepackage{graphicx}
\usepackage{url}
\usepackage[section]{placeins}
\usepackage{tikz}
\usetikzlibrary{arrows,calc,decorations.markings,math,arrows.meta}
\usepackage{rotating}
\usepackage[percent]{overpic}
\usepackage[utf8]{inputenc}
\usepackage{xcolor}
\usepackage{pgfplots}
\usetikzlibrary{pgfplots.groupplots}
\usepackage{listings}
\usepackage{matlab-prettifier}
\usepackage{siunitx}
\definecolor{szary}{rgb}{0.95,0.95,0.95}
\sisetup{detect-weight,exponent-product=\cdot,output-decimal-marker={,},per-mode=symbol,binary-units=true,range-phrase={-},range-units=single}
\graphicspath{{wykresy/}}
%konfiguracje pakietu listings
\lstset{
	backgroundcolor=\color{szary},
	frame=single,
	breaklines=true,
}
\lstdefinestyle{customlatex}{
	basicstyle=\footnotesize\ttfamily,
	%basicstyle=\small\ttfamily,
}
\lstdefinestyle{customc}{
	breaklines=true,
	frame=tb,
	language=C,
	xleftmargin=0pt,
	showstringspaces=false,
	basicstyle=\small\ttfamily,
	keywordstyle=\bfseries\color{green!40!black},
	commentstyle=\itshape\color{purple!40!black},
	identifierstyle=\color{blue},
	stringstyle=\color{orange},
}
\lstdefinestyle{custommatlab}{
	captionpos=t,
	breaklines=true,
	frame=tb,
	xleftmargin=0pt,
	language=matlab,
	showstringspaces=false,
	%basicstyle=\footnotesize\ttfamily,
	basicstyle=\scriptsize\ttfamily,
	keywordstyle=\bfseries\color{green!40!black},
	commentstyle=\itshape\color{purple!40!black},
	identifierstyle=\color{blue},
	stringstyle=\color{orange},
}

%wymiar tekstu (bez �ywej paginy)
\textwidth 160mm \textheight 247mm

%ustawienia pakietu pgfplots
\pgfplotsset{
tick label style={font=\scriptsize},
label style={font=\small},
legend style={font=\small},
title style={font=\small}
}

\def\figurename{Rys.}
\def\tablename{Tab.}

%konfiguracja liczby p�ywaj�cych element�w
\setcounter{topnumber}{0}%2
\setcounter{bottomnumber}{3}%1
\setcounter{totalnumber}{5}%3
\renewcommand{\textfraction}{0.01}%0.2
\renewcommand{\topfraction}{0.95}%0.7
\renewcommand{\bottomfraction}{0.95}%0.3
\renewcommand{\floatpagefraction}{0.35}%0.5

\begin{document}
\frenchspacing
\pagestyle{uheadings}

%strona tytu�owa
\title{\bf Sprawozdanie z projektu nr 1\vskip 0.1cm}
\author{Kamil Gabryjelski}
\date{2017}

\makeatletter
\renewcommand{\maketitle}{\begin{titlepage}
\begin{center}{\LARGE {\bf
Wydział Elektroniki i Technik Informacyjnych}}\\
\vspace{0.4cm}
{\LARGE {\bf Politechnika Warszawska}}\\
\vspace{0.3cm}
\end{center}
\vspace{5cm}
\begin{center}
{\bf \LARGE Algorytmy ewolucyjne \vskip 0.1cm}
\end{center}
\vspace{1cm}
\begin{center}
{\bf \LARGE \@title}
\end{center}
\vspace{2cm}
\begin{center}
{\bf \Large \@author \par}
\end{center}
\vspace*{\stretch{6}}
\begin{center}
\bf{\large{Warszawa, \@date\vskip 0.1cm}}
\end{center}
\end{titlepage}
}
\makeatother

\maketitle

\tableofcontents
\chapter{Opis funkcji}
Parametry wygenerowano przy pomocy wzoru:
\begin{equation}
Int[4*rand()]/2
\end{equation}
gdzie $rand()$ to generator liczb losowych z rozkładem równomiernym z zakresu $<-1,1>$.
Wylosowane parametry mają wartości: $a=-1,5$ oraz $b=-1$. Funkcja Rosenbrocka ma zatem postać:
\begin{equation}
\begin{split}
f(x)=(1-x-1,5)^2+100*(y+1-(x+1,5)^2)^2
\end{split}
\end{equation}
Punkty startowe generowane były przy pomocy wzoru:
\begin{equation}
\begin{split}
x=a+2*rand() \\
y=b+2*rand()
\end{split}
\end{equation}
Wylosowane punkty mają następujące wartości:
\begin{itemize}
  \item $x=-1,12131$, $y=-0,23678$
  \item $x=-2,60771$, $y=-0,38339$
  \item $x=0,31282$, $y=-0,01969$
  \item $x=-2,97363$, $y=-2,46108$
\end{itemize}
Funkcja Rosenbrocka o podanych parametrach przedstawiona jest na wykresie konturowym \ref{fig:rosenbrock}.
\begin{figure}
  \centering
  \includesvg{wykresy/rosenbrock}
  \caption{Wykres funkcji Rosenbrocka o parametrach $a=-1.5$ i $b=-1$}
  \label{fig:rosenbrock}
\end{figure}
\chapter{Metody optymalizacji}
W celu znalezienia minimum funkcji Rosenbrocka użyto czterech algorytmów optymalizacji: Neldera-Meada, Powella, gradientów sprzężonych oraz Newtona. Wyniki symulacji opisane zostały w kolejnych podrozdziałach.

\section{Metoda Neldera-Meada}
Metoda Neldera-Meada dla każdego punktu startowego zminimalizowała funkcję Rosenbrocka z precyzją rzędu $10^{-10}$, co jest zadowalającym wynikiem. Algorytm ten jest jednak wolniejszy od pozostałych porównywanych metod. Dla punktu startowego $(-2,60771, -0,38339)$ potrzebował aż 95 iteracji, aby znaleźć minimum.

Wyniki symulacji przedstawiają wykresy \ref{fig:nelder-mead_0}, \ref{fig:nelder-mead_1}, \ref{fig:nelder-mead_2} oraz \ref{fig:nelder-mead_3}.

\begin{figure}
  \centering
  \includesvg{nelder-mead_0}
  \caption{Optymalizacja metodą Neldera-Meada dla punktu startowego $x=-1,12131$, $y=-0,23678$}
  \label{fig:nelder-mead_0}
\end{figure}

\begin{figure}
  \centering
  \includesvg{nelder-mead_1}
  \caption{Optymalizacja metodą Neldera-Meada dla punktu startowego $x=-2,60771$, $y=-0,38339$}
  \label{fig:nelder-mead_1}
\end{figure}

\begin{figure}
  \centering
  \includesvg{nelder-mead_2}
  \caption{Optymalizacja metodą Neldera-Meada dla punktu startowego $x=0,31282$, $y=-0,01969$}
  \label{fig:nelder-mead_2}
\end{figure}

\begin{figure}
  \centering
  \includesvg{nelder-mead_3}
  \caption{Optymalizacja metodą Neldera-Meada dla punktu startowego $x=-2,97363$, $y=-2,46108$}
  \label{fig:nelder-mead_3}
\end{figure}


\section{Metoda Powella}
Metoda Powella potrzebowała mniej iteracji od metody Neldera-Meada, osiągając na ogół precyzję rzędu nawet $10^{-30}$. Dla punktu startowego $(-2,60771, -0,38339)$ okazała się nieskuteczna, przekraczając limit ewaluacji funkcji. Mimo niepowodzenia, algorytm zminimalizował funkcję Rosenbrocka z tolerancją rzędu $10^{-6}$, co jest akceptowalnym wynikiem. Poza wspomnianym przypadkiem, w którym metoda zawiodła, do optymalizacji wymagane jest w najgorszym przypadku zaledwie 14 iteracji. Można zatem stwierdzić, że algorytm Powella jest najszybszym spośród testowanych.

Wyniki symulacji przedstawiają wykresy \ref{fig:powell_0}, \ref{fig:powell_1}, \ref{fig:powell_2} oraz \ref{fig:powell_3}.

\begin{figure}
  \centering
  \includesvg{powell_0}
  \caption{Optymalizacja metodą Powella dla punktu startowego $x=-1,12131$, $y=-0,23678$}
  \label{fig:powell_0}
\end{figure}

\begin{figure}
  \centering
  \includesvg{powell_1}
  \caption{Optymalizacja metodą Powella dla punktu startowego $x=-2,60771$, $y=-0,38339$}
  \label{fig:powell_1}
\end{figure}

\begin{figure}
  \centering
  \includesvg{powell_2}
  \caption{Optymalizacja metodą Powella dla punktu startowego $x=0,31282$, $y=-0,01969$}
  \label{fig:powell_2}
\end{figure}

\begin{figure}
  \centering
  \includesvg{powell_3}
  \caption{Optymalizacja metodą Powella dla punktu startowego $x=-2,97363$, $y=-2,46108$}
  \label{fig:powell_3}
\end{figure}


\section{Metoda gradientów sprzężonych}
Dla punktów $(0,31282, -0,01969)$ i $(-2,97363, -2,46108)$ metoda gradientów sprzężonych była nieskuteczna, kończąc swoje działanie już po kilku iteracjach. W pozostałych dwóch przypadkach algorytm znajdował minimum funkcji Rosenbrocka z precyzją rzędu $10^{-12}$, do czego potrzebował około 20 iteracji. Można zatem stwierdzić, że metoda gradientów sprzężonych jest szybka, ale powodzenie jej działania zależy od wyboru punktu początkowego. 

Wyniki symulacji przedstawiają wykresy \ref{fig:cg_0}, \ref{fig:cg_1}, \ref{fig:cg_2} oraz \ref{fig:cg_3}.

\begin{figure}
  \centering
  \includesvg{cg_0}
  \caption{Optymalizacja metodą gradientów sprzężonych dla punktu startowego $x=-1,12131$, $y=-0,23678$}
  \label{fig:cg_0}
\end{figure}

\begin{figure}
  \centering
  \includesvg{cg_1}
  \caption{Optymalizacja metodą gradientów sprzężonych dla punktu startowego $x=-2,60771$, $y=-0,38339$}
  \label{fig:cg_1}
\end{figure}

\begin{figure}
  \centering
  \includesvg{cg_2}
  \caption{Optymalizacja metodą gradientów sprzężonych dla punktu startowego $x=0,31282$, $y=-0,01969$}
  \label{fig:cg_2}
\end{figure}

\begin{figure}
  \centering
  \includesvg{cg_3}
  \caption{Optymalizacja metodą gradientów sprzężonych dla punktu startowego $x=-2,97363$, $y=-2,46108$}
  \label{fig:cg_3}
\end{figure}


\section{Metoda Newtona}
Metoda Newtona dla każdego punktu startowego zminimalizowała funkcję Rosenbrocka z precyzją rzędu $10^{-11}$, co jest podobnym rezultatem do osiągniętego przez metodę Neldera-Meada. Algorytm Newtona jest jednak szybszy, potrzebując w najgorszym przypadku 47 iteracji do osiągnięcia zadowalającej tolerancji. Można również stwierdzić, że metoda Newtona jest lepsza od metody gradientów sprzężonych ze względu na jej niezawodność. 

Wyniki symulacji przedstawiają wykresy \ref{fig:newton_0}, \ref{fig:newton_1}, \ref{fig:newton_2} oraz \ref{fig:newton_3}.

\begin{figure}
  \centering
  \includesvg{newton_0}
  \caption{Optymalizacja metodą Newtona dla punktu startowego $x=-1,12131$, $y=-0,23678$}
  \label{fig:newton_0}
\end{figure}

\begin{figure}
  \centering
  \includesvg{newton_1}
  \caption{Optymalizacja metodą Newtona dla punktu startowego $x=-2,60771$, $y=-0,38339$}
  \label{fig:newton_1}
\end{figure}

\begin{figure}
  \centering
  \includesvg{newton_2}
  \caption{Optymalizacja metodą Newtona dla punktu startowego $x=0,31282$, $y=-0,01969$}
  \label{fig:newton_2}
\end{figure}

\begin{figure}
  \centering
  \includesvg{newton_3}
  \caption{Optymalizacja metodą Newtona dla punktu startowego $x=-2,97363$, $y=-2,46108$}
  \label{fig:newton_3}
\end{figure}
\chapter{Statystyki}
Tabela \ref{stats} przedstawia porównanie metod optymalizacji dla czterech losowych punktów startowych.
\begin{table}[]
\centering
\caption{Porównanie metod optymalizacji}
\label{stats}
\begin{tabular}{|c|c|c|c|c|}
\hline
Punkt startowy                         & Metoda                 & Punkt końcowy         & Dokładność                  & Liczba iteracji \\ \hline
\multirow{4}{*}{\makecell{$x=-1,12131$ \\ $y=-0,23678$}} & Neldera-Meada          & \makecell{$x=-0,50003$ \\ $y=-0,00005$} & $8,03722^{-10}$ & $67$           \\ \cline{2-5} 
                                       & Powella                & \makecell{$x=-0,50000$ \\ $y=0,00000$}  & 0                           & $9$               \\ \cline{2-5} 
                                       & Gradientów sprzężonych & \makecell{$x=-0,50000$ \\ $y=0,00000$}  & $2,52116^{-11}$ & $28$              \\ \cline{2-5} 
                                       & Newtona                & \makecell{$x=-0,50000$ \\ $y=0,00000$}  & $2,22805^{-13}$ & $24$              \\ \hline
\multirow{4}{*}{\makecell{$x=-2,60771$ \\ $y=-0,38339$}} & Neldera-Meada          & \makecell{$x=-0,50002$ \\ $y=-0,00004$} & $6,47764^{-10}$ & $97$              \\ \cline{2-5} 
                                       & Powella                & \makecell{$x=-0,50000$ \\ $y=0,00000$}  & $1,04149^{-27}$ & $10$              \\ \cline{2-5} 
                                       & Gradientów sprzężonych & \makecell{$x=-0,50000$ \\ $y=0,00000$}  & $8,96869^{-13}$ & $26$              \\ \cline{2-5} 
                                       & Newtona                & \makecell{$x=-0,50001$ \\ $y=-0,00001$} & $4,68967^{-11}$ & $48$           \\ \hline
\multirow{4}{*}{\makecell{$x=0,31282$ \\ $y=-0,01969$}}  & Neldera-Meada          & \makecell{$x=-0,50001$ \\ $y=-0,00001$} & $3,33663^{-10}$ & $54$              \\ \cline{2-5} 
                                       & Powella                & \makecell{$x=-0,50031$ \\ $y=0,00088$}  & $6,95587^{-6}$  & $74$              \\ \cline{2-5} 
                                       & Gradientów sprzężonych & \multicolumn{3}{c|}{Niepowodzenie}                                    \\ \cline{2-5} 
                                       & Newtona                & \makecell{$x=-0,50000$ \\ $y=0,00000$}  & $4,12421^{-12}$ & $10$              \\ \hline
\multirow{4}{*}{\makecell{$x=-2,97363$ \\ $y=-2,46108$}} & Neldera-Meada          & \makecell{$x=-0,50000$ \\ $y=0,00001$}  & $1,75344^{-9}$  & $78$              \\ \cline{2-5} 
                                       & Powella                & \makecell{$x=-0,50000$ \\ $y=0,00000$}  & $5,71924^{-30}$ & $14$              \\ \cline{2-5} 
                                       & Gradientów sprzężonych & \multicolumn{3}{c|}{Niepowodzenie}                                    \\ \cline{2-5} 
                                       & Newtona                & \makecell{$x=-0,50003$ \\ $y=0,00006$}  & $5,98328^{-13}$ & $37$              \\ \hline
\end{tabular}
\end{table}
\end{document}

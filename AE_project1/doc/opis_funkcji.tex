\chapter{Opis funkcji}
Parametry wygenerowano przy pomocy wzoru:
\begin{equation}
Int[4*rand()]/2
\end{equation}
gdzie $rand()$ to generator liczb losowych z rozkładem równomiernym z zakresu $<-1,1>$.
Wylosowane parametry mają wartości: $a=-1,5$ oraz $b=-1$. Funkcja Rosenbrocka ma zatem postać:
\begin{equation}
\begin{split}
f(x)=(1-x-1,5)^2+100*(y+1-(x+1,5)^2)^2
\end{split}
\end{equation}
Punkty startowe generowane były przy pomocy wzoru:
\begin{equation}
\begin{split}
x=a+2*rand() \\
y=b+2*rand()
\end{split}
\end{equation}
Wylosowane punkty mają następujące wartości:
\begin{itemize}
  \item $x=-1,12131$, $y=-0,23678$
  \item $x=-2,60771$, $y=-0,38339$
  \item $x=0,31282$, $y=-0,01969$
  \item $x=-2,97363$, $y=-2,46108$
\end{itemize}
Funkcja Rosenbrocka o podanych parametrach przedstawiona jest na wykresie konturowym \ref{fig:rosenbrock}.
\begin{figure}
  \centering
  \includesvg{wykresy/rosenbrock}
  \caption{Wykres funkcji Rosenbrocka o parametrach $a=-1.5$ i $b=-1$}
  \label{fig:rosenbrock}
\end{figure}